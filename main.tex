% Welcome! This is the unofficial University of Udine beamer template.

% See README.md for more informations about this template.

% This style has been developed following the "Manuale di Stile"
% (Style Manual) of the University of Udine. You can find the
% manual here: https://www.uniud.it/it/ateneo-uniud/ateneo-uniud/identita-visiva/manuali-immagine-stile/manuale-stile

% Note: for some reason, the RGB values specified in the manual
% do NOT render correctly in Beamer, so they have been redefined
% for this document using the high level chromo-optic deep neural 
% quantistic technology offered by Microsoft Paint's color picker.

% We defined four theme colors: UniBrown, UniBlue, UniGold
% and UniOrange. For example, to write some uniud-brownish
% text, just use: \textcolor{UniBrown}{Hello!}

% Note that [usenames,dvipsnames] is MANDATORY due to compatibility
% issues between tikz and xcolor packages.

\documentclass[usenames,dvipsnames]{beamer}
\usepackage[utf8]{inputenc}
\usepackage{verbatim}
\usetheme{uniud}

%%% Bibliography
\usepackage[style=authoryear,backend=biber]{biblatex}
\addbibresource{bibliography.bib}

% Author names in publication list are consistent 
% i.e. name1 surname1, name2 surname2
% See https://tex.stackexchange.com/questions/106914/biblatex-does-not-reverse-the-first-and-last-names-of-the-second-author
\DeclareNameAlias{author}{first-last}

%%% Suppress biblatex annoying warning
\usepackage{silence}
\WarningFilter{biblatex}{Patching footnotes failed}

%%% Some useful commands
% pdf-friendly newline in links
\newcommand{\pdfnewline}{\texorpdfstring{\newline}{ }} 
% Fill the vertical space in a slide (to put text at the bottom)
\newcommand{\framefill}{\vskip0pt plus 1fill}

\title[Universitat de Girona]{Word-Processing-based Routing\\for Cayley Graphs}
\date[Apr 2018]{Apr 5, 2018}
\author[Daniela Aguirre Guerrero]{
  Daniela Aguirre Guerrero
  \pdfnewline
  \texttt{d.aguirre@correo.ler.uam.mx}
}
\institute{Universitat de Girona}

\begin{document}
\begin{frame}
\titlepage
\end{frame}

\begin{frame}{Outline}
\tableofcontents
\end{frame}

\section{Word processing in Cayley graphs}
\begin{frame}{\textit{ShortLex} Automatic Groups (\texttt{SAG}) (i)}

\begin{definition}[$k$-fellow-traveler property]
Two paths $\widehat{u}$ and $\widehat{v}$ in  $\Gamma(\mathcal{G},S)$ are said to satisfy the $k$-fellow-traveler property if they are at distance $k$, i.e. $d_\Gamma(\widehat{u},\widehat{v})$.
\end{definition}

\begin{definition}[\textit{ShortLex} Automatic Group]
Let $\mathcal{G}=\left\langle S\right\rangle$ be a \texttt{SAG}, then every pair of \textit{shortlex} paths in $\Gamma(\mathcal{G},S)$ that begin at the same node and whose end nodes are adjacent satisfy the $k$-fellow-travel property.
\end{definition}

\end{frame}

\begin{frame}{\textit{ShortLex} Automatic Groups (\texttt{SAG}) (ii)}

\begin{fact}
Let $\mathcal{G}=\left\langle S\right\rangle$ be a \texttt{SAG} with traveler constant $k$. Then, the geodesic paths between any pair of nodes in $\Gamma(\mathcal{G},S)$ are uniformly separated at distance $k$.
\end{fact}

\end{frame}

\begin{frame}{The Word-difference Automaton (\texttt{WDA}) (i)}

\begin{definition}[\texttt{WDs}]
The set of Word-differences (\texttt{WDs}) of $\Gamma(\mathcal{G},S)$ is given by the labels of the nodes in $N(\Gamma,e_\mathcal{A})$, i.e.
$$\mathbb{D}=\{w\in L:|w|\leq k\}.$$
\end{definition}

\end{frame}

\begin{frame}{The Word-difference Automaton (\texttt{WDA}) (ii)}

\begin{definition}[\texttt{WDA}]
The \texttt{WDA} is a $2$-variable \texttt{FSA} over $\mathcal{A}$ denoted by \texttt{diff} and with state set given by $\mathbb{D}$. It accepts $(r,t)$, if and only if
$\widehat{r}$ and $\widehat{t}$ satisfy that
\begin{enumerate}
    \item $d(\widehat{r},\widehat{t})\leq k$, and
    \item $q^{(r(i),t(i))}\in \mathbb{D}$ is the \textit{shortLex} path from $\overline{r(i)}$ to $\overline{t(i)}$. 
\end{enumerate}
\end{definition}

\end{frame}
\section{Path computation algorithms}
\begin{frame}{Computing the shortest path (i)}

Let $(r,t)$ be a tuple recognized by \textit{diff}, such that $r<_\mathcal{A} t$ and $q^{(r,t)}=e_\mathcal{A}$. Then paths $\widehat{r}$ and $\widehat{t}$ beginning at the same node satisfy that:
\begin{enumerate}
    \item $\widehat{r}$ and $\widehat{t}$ join the same pair of nodes.
    \item $|\widehat{r}|\leq|\widehat{t}|$.
\end{enumerate}
\end{frame}


\begin{frame}{Computing the minimal paths (i)}

Let $(r,t)$ be a tuple recognized by \textit{diff}, such that $t\in L$, $r<_\mathcal{A}t$ and $q^{(r,t)}=e_\mathcal{A}$. Then paths $\widehat{r}$ and $\widehat{t}$ beginning at the same node satisfy that:
\begin{enumerate}
    \item $\widehat{r}$ and $\widehat{t}$ join the same pair of nodes.
    \item $\widehat{r}$ and $\widehat{t}$ are minimal paths.
    \item $d_\Gamma(\widehat{r},\widehat{t})\leq k$.
\end{enumerate}
\end{frame}

\begin{frame}{Computing the minimal paths (ii)}

\begin{lemma}[The set of minimal paths]
Let $\widehat{w}$ be the \textit{shortLex} path between two nodes $\overline{u}$ to $\overline{v}$. The set of minimal paths from $\overline{u}$ to $\overline{v}$ that are at uniform distance $jk$ from $\widehat{w}$ is given by
$$\mathbb{P}_{j}(w) =\{t\in \mathcal{F}(\mathcal{A}):\exists r \in \mathbb{P}_{j-1}(w),\textnormal{ s.t. } q^{(r,t)}\in e_\mathcal{A}\},$$

where $j>0$ and $\mathbb{P}_0=\{w\}$.
\end{lemma}

\end{frame}

\begin{frame}{Computing the minimal paths (iii)}

\begin{corollary}[The set of minimal paths]
Let $\widehat{w}$ be the \textit{shortLex} path from $\overline{u}$ to $\overline{v}$. The set of all the minimal paths from $\overline{u}$ to $\overline{v}$ is given by $\mathbb{P}_{\lceil D_\Gamma / k\rceil}$.
\end{corollary}

\end{frame}

\begin{frame}{Computing all the paths (i)}

Let $(r,t)$ be a tuple recognized by \textit{diff}, such that $r<_\mathcal{A}t$. Then paths $\widehat{r}$ and $\widehat{t}$ beginning at the same node satisfy that:
\begin{enumerate}
    \item $\widehat{t}$ and $\widehat{rq^{(r,t)}}$ join the same pair of nodes.
    \item  $|rq^{(r,t)}| = |r|+|q^{(r,t)}|$, then $|\widehat{t}|\leq|\widehat{rq^{(r,t)}}| \leq |\widehat{t}|+k$.
\end{enumerate}
\end{frame}

\begin{frame}{Computing all the paths (ii)}

\begin{lemma}[The paths of bounded length]
Let $\widehat{w}$ be the \textit{shortLex} path between two nodes $\overline{u}$ to $\overline{v}$. The set of paths from $\overline{u}$ to $\overline{v}$ with length at most $d_\Gamma(\overline{u},\overline{v})+ik$ is given by
$$[w]_i =\{(rq^{(r,t)})_{red}\in \mathcal{F}(\mathcal{A}):\forall t \in [w]_{i-1}\},$$

where $i>0$ and $[w]_0=\mathbb{P}_{\lceil D_\Gamma / k\rceil}(w)$.
\end{lemma}

\end{frame}

\begin{frame}{Computing all the paths (iii)}

\begin{corollary}
[The set of all  paths]
Let $\widehat{w}$ be the \textit{shortLex} path between two nodes $\overline{u}$ to $\overline{v}$. The set of all paths from $\overline{u}$ to $\overline{v}$ are given by the words in $[w]_{\lceil D_\Gamma / k\rceil}$.
\end{corollary}

\end{frame}
\input{sections/sample}
\end{document}